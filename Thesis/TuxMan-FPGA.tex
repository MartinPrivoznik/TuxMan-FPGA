\documentclass{report}
\usepackage{ucs} 
\usepackage[utf8x]{inputenc} 	
\usepackage[czech]{babel}
\usepackage{pdfpages}
\usepackage{hyperref}
\usepackage{graphicx}
\usepackage{datetime}
\usepackage{titlesec}
\newcommand*\wildcard[2][5cm]{\vspace*{2cm}\parbox{#1}{\hrulefill\par#2}}  

\title{TuxMan - FPGA}
\titleformat{\paragraph}
\date{11/4/2019}
\author{Martin Přívozník}

\begin{document}
  \includepdf[page={1}]{Front-Page}
  \newpage
  \pagenumbering{gobble}

 \section*{Anotace (Resumé)}
 Tématem práce je návrh arkádové hry na RTL úrovni a její implementace na programovatelném hradlovém poli, tj. FPGA. Uživatelský vstup je zajištěn pomocí PS/2 klávesnice a výstup prostřednictvím VGA. K implementaci je použit jazyk VHDL.

\section*{Klíčová slova}
RTL, programovatelné hradlové pole, FPGA, PS/2, VGA, VHDL

 \section*{Summary}
 The topic of this thesis is designing an arcade game on an RTL level and its implementation on a programmable gate array, i.e. FPGA. User input is provided by PS/2 keyboard and output is shown using VGA. Language called VHDL is used for implementation.

\section*{Keywords}
RTL, programmable gate array, FPGA, PS/2, VGA, VHDL

 \newpage
\vspace*{\fill}

 \section*{Čestné prohlášení}
 Prohlašuji, že jsem předkládanou maturitní/ročníkovou práci vypracoval sám a uvedl jsem veškerou použitou literaturu a bibliografické citace.

\vspace{ 2cm}
  V Liberci dne \today
  \hspace{2cm}
  \wildcard{Martin Přívozník}

\newpage
\tableofcontents

\newpage
\pagenumbering{arabic}

\chapter*{Úvod}
\addcontentsline{toc}{section}{Úvod}

\chapter{Analýza}
Tato kapitola obsahuje stručný souhrn znalostí a informací potřebných pro následný návrh a implementaci. V sekci~\ref{sec:cislicovynavrh} je vysvětlen číslicový obvod a postup jeho návrhut. V sekci~\ref{sec:fpga} stručně vysvětluji programovatelné hradlové pole a dále vybranou vývojovou desku. Sekce~\ref{sec:rozhrani} se zabývá použitými komunikačními rozhraními, které zajišťují uživatelský vstup a výstup. Na závěr kapitoli, v sekci~\ref{sec:hrapacman} vysvětluji princip a funkčnost hry, jíž je vzorem pro můj návrh.

\section{Číslicový návrh}\label{sec:cislicovynavrh}
V této sekci se věnuji tomu, co je číslicový obvod a jak jej navrhnout jak ve schématu, tak v jazyce popisujícím hardware. V podsekci~\ref{sec:kombinacniobvody} rozeberu logické funkce, prostředky jejich popisu a realizace pomocí logických hradel. Podsekce~\ref{sec:sekvencniobvody}  je zaměřená na návrh sekvenčních obvodů a synchronních sekvenčních automatů (FSM), na což naváže podsekce~\ref{sec:synchronniaasynchronninavrh}, ve které vysvěluji princip hodinových domén a plně sekvenčního návrhu. V podsekci~\ref{sec:jazykvhdl} stručně ukážu, jak převést schéma číslicového obvodu do kódu v jazyce popisujícím hardware (Hardware Description Language, HDL), v mém případě do jazyka Very High Speed Integrated Circuit Hardware Description Language (VHDL)

\subsection{Kombinační obvody}\label{sec:kombinacniobvody}

\subsubsection{Booleovská funkce}
Booleovská funkce je funkce $N$ vstupů a $M$ výstupů nad množinou $\{0, 1\}$. V případě, kdy má funkce více jak jeden výstup, lze ji rozdělit na $M$ funkcí s jedním výstupem. Pro její reprezentaci můžeme využít kombinatorické struktury, které funkci vyjadřují. Příkladem reprezentace je pravdivostní tabulka, viz. tabulka~\ref{tab:andtable}, kde $in_1$ a $in_2$ jsou vstupní hodnoty a $out$ je výstupní.
\begin{table}[!htbp]
\centering
\begin{tabular}{ |c|c|c| } 
   	\hline
	$in_1$ & $in_2$ & $out$ \\
   	\hline
	$0$ & $0$ & $f(0,0)$ \\
   	\hline
	$0$ & $1$ & $f(0,1)$ \\
   	\hline
	$1$ & $0$ & $f(1,0)$ \\
   	\hline
	$1$ & $1$ & $f(1,1)$ \\
   	\hline
\end{tabular}
    \caption{Pravdivostní tabulka.}
    \label{tab:andtable}
\end{table}
Tato tabulka obsahuje vždy $N^2$ řádků, aby reprezentovala výstupní hodnotu pro všechny možné kombinace vstuních hodnot. Dalším podstatným příkladem je Booleovská formule. K vyjádření formule a zároveň k popisu booleovské funkce používáme nejčastěji tyto základní funkce, viz. tabulka~\ref{tab:logickefunkce}

\begin{table}[!htbp]
\centering
\begin{tabular}{ |c|c|c| } 
   	\hline
	Název & Pravdivostní tabulka & Formule \\
   	\hline
	AND (logický součin) & \begin{tabular}{ |c|c|c| } 
	   	\hline
		$in_1$ & $in_2$ & $out$ \\
	   	\hline
		$0$ & $0$ & $0$ \\
	   	\hline
		$0$ & $1$ & $0$ \\
	   	\hline
		$1$ & $0$ & $0$ \\
	   	\hline
		$1$ & $1$ & $1$ \\
	   	\hline
	\end{tabular} & $out = in_1*in_2$ \\
   	\hline
	NAND (negovaný logický součin) & \begin{tabular}{ |c|c|c| } 
	   	\hline
		$in_1$ & $in_2$ & $out$ \\
	   	\hline
		$0$ & $0$ & $1$ \\
	   	\hline
		$0$ & $1$ & $1$ \\
	   	\hline
		$1$ & $0$ & $1$ \\
	   	\hline
		$1$ & $1$ & $0$ \\
	   	\hline
	\end{tabular} & $out = \overline{in_1*in_2}$ \\
	\hline
	OR (logický součet) & \begin{tabular}{ |c|c|c| } 
	   	\hline
		$in_1$ & $in_2$ & $out$ \\
	   	\hline
		$0$ & $0$ & $0$ \\
	   	\hline
		$0$ & $1$ & $1$ \\
	   	\hline
		$1$ & $0$ & $1$ \\
	   	\hline
		$1$ & $1$ & $1$ \\
	   	\hline
	\end{tabular} & $out = in_1+in_2$ \\
	\hline
	NOR (negovaný logický součet) & \begin{tabular}{ |c|c|c| } 
	   	\hline
		$in_1$ & $in_2$ & $out$ \\
	   	\hline
		$0$ & $0$ & $1$ \\
	   	\hline
		$0$ & $1$ & $0$ \\
	   	\hline
		$1$ & $0$ & $0$ \\
	   	\hline
		$1$ & $1$ & $0$ \\
	   	\hline
	\end{tabular} & $out = \overline{in_1+in_2}$ \\
	\hline
	NOT (logická negace) & \begin{tabular}{ |c|c| } 
	   	\hline
		$in_1$ & $out$ \\
	   	\hline
		$0$ & $1$ \\
	   	\hline
		$1$ & $0$\\
	   	\hline
	\end{tabular} & $out = \overline{in_1}$ \\
	\hline
	YES (opakovač) & \begin{tabular}{ |c|c| } 
	   	\hline
		$in_1$ & $out$ \\
	   	\hline
		$0$ & $0$ \\
	   	\hline
		$1$ & $1$\\
	   	\hline
	\end{tabular} & $out = in_1$ \\
	\hline
	XOR (exkluzivní logický součet) & \begin{tabular}{ |c|c|c| } 
	   	\hline
		$in_1$ & $in_2$ & $out$ \\
	   	\hline
		$0$ & $0$ & $0$ \\
	   	\hline
		$0$ & $1$ & $1$ \\
	   	\hline
		$1$ & $0$ & $1$ \\
	   	\hline
		$1$ & $1$ & $0$ \\
	   	\hline
	\end{tabular} & $out = in_1 \oplus in_2$ \\
	\hline
	XNOR (negovaný exkluzivní logický součet) & \begin{tabular}{ |c|c|c| } 
	   	\hline
		$in_1$ & $in_2$ & $out$ \\
	   	\hline
		$0$ & $0$ & $1$ \\
	   	\hline
		$0$ & $1$ & $0$ \\
	   	\hline
		$1$ & $0$ & $0$ \\
	   	\hline
		$1$ & $1$ & $1$ \\
	   	\hline
	\end{tabular} & $out =\overline{in_1 \oplus in_2}$ \\
   	\hline
\end{tabular}
    \caption{Tabulka nejpoužívanějších základních logických funkcí.}
    \label{tab:logickefunkce}
\end{table}
Uvážíme-li Booleovu algebru, platí pro OR a AND následující (tabulka~\ref{tab:pravidla})
\begin{table}[h]
\centering
\begin{tabular}{ c c c } 
	$a+ b = b + a$ & $a*b = b*a$ & (komutativita) \\
	$a+(b+c) = (a+b) + c$ & $a*(b*c) = (a*b)*c$ & (asociativita) \\
	$a+(b*c) = (a+b) * (a+c)$ & $a*(b+c) = (a*b) + (a*c)$ & (distributivita) \\
	$a+0 = a$ & $a*1 = a$ & (neutralita 0 a 1) \\
	$a + \overline{a} = 1$ & $a * \overline{a} = 0$ & (vlastnosti negace) \\
\end{tabular}
    \caption{Tahle tabulka by měla být pod textem "platí pro OR a AND následující:".}
    \label{tab:pravidla}
\end{table}


\subsection{Sekvenční obvody}\label{sec:sekvencniobvody}
\subsection{Synchronní a asynchronní návrh}\label{sec:synchronniaasynchronninavrh}
\subsection{Jazyk VHDL}\label{sec:jazykvhdl}

\section{Programovatelná hradlová pole FPGA}\label{sec:fpga}
Tato sekce se zabývá tím, co jsou programovatelná hradlová pole (Field of programmable gate array, FPGA) a jak probíhá práce s těmito hradlovými poli na vývojové desce. V podsekci~\ref{sec:dostupneprostredky} popisuji části FPGA, které jsou použity pro implementaci číslicového obvodu. Podsekce~\ref{sec:logickasynteza} stručně vysvětluje kroky nezbytné pro implementaci samotného číslicového obvodu na základě jeho popisu VHDL kódem. V podsekcích \ref{sec:basys2} a \ref{sec:xilinxise} je stručně popsána použitá vývojová deska a vývojové prostředí, které se váže k FPGA, které deska obsahuje.

\subsection{Dostupné prostředky}\label{sec:dostupneprostredky}
\subsection{Logická syntéza}\label{sec:logickasynteza}
\subsection{Vývojová deska Digilent Basys 2}\label{sec:basys2}
\subsection{Prostředí Xilinx ISE}\label{sec:xilinxise}

\section{Rozhraní}\label{sec:rozhrani}
Tato sekce popisuje komunikační rozhraní použitá v návrhu a implementaci, která zajišťují uživatelský vstup a výstup. V podsekci~\ref{sec:7segment} vysvětluji rozhraní 7-segmentového displeje. Sekce  \ref{sec:ps2} a \ref{sec:vga} popisují protokoly PS/2 a VGA.
\subsection{7-segmentový displej}\label{sec:7segment}
\subsection{PS/2}\label{sec:ps2}
\subsection{VGA}\label{sec:vga}

\section{Hra Pacman}\label{sec:hrapacman}
Tato sekce vysvětluje hru, již má můj návrh klonovat. Podsekce~\ref{sec:principhry} popisuje cíl a pravidla hry. V podsekci~\ref{sec:ovladanihry} vysvětluji ovládání hry.
\subsection{Princip}\label{sec:principhry}
\subsection{Ovládání}\label{sec:ovladanihry}

\chapter{Návrh hry}
V této kapitole se zabývám návrhem herní logiky, na základě které je navržen číslicový obvod pro finální implementaci. Sekce~\ref{sec:specifikacehry} rozebírá konkrétní prvky hry a návrhy pro jejich řešení. V sekci~\ref{sec:hernitextury} vysvětluji, jak jsou navržené textury, které se zobrazují na výstup.

\section{Specifikace hry}\label{sec:specifikacehry}
V této sekci rozebírám konkrétní prvky hry a vysvětluji návrhy pro jejich řešení. V podsekci~\ref{sec:cilachovani} popisuji, jak má hra fungovat jako celek a jak dosáhnout cíle hry. Podsekce~\ref{sec:mapajakomrizka} vysvětluje návrh herního pole, na kterém se hra odehrává a sekce~\ref{sec:postavyachovani} poté popisuje, jak se chovají jednotlivé postavy na navrženém herním poli.

\subsection{Cíl a chování hry}\label{sec:cilachovani}
\subsection{Herní mapa}\label{sec:mapajakomrizka}
\subsection{Postavy a jejich chování}\label{sec:postavyachovani}

\section{Herní textury}\label{sec:hernitextury}
Tato sekce popisuje návrh textur, které se zobrazují na výstup. V podsekci~\ref{sec:RGB} vysvětluji způsob použití barev pro výstup. Sekce~\ref{sec:texturyjakomatice} se zabývá tím, v jaké podobě jsou textury, které se mají propisovat. 
\subsection{RGB}\label{sec:RGB}
\subsection{Textury jako matice barev}\label{sec:texturyjakomatice}

\chapter{Implementace}
Tato kapitola se zabývá návrhem číslicových obvodů pro jednotlivé logicky oddělené bloky na základě herní logiky a jejich implementací. Sekce \ref{sec:ctenizps2} a \ref{sec:vystupnavga} popisují číslicové obvody navržené pro čtení vstupních dat z klávesnice a generování výstupu na monitor. Sekce~\ref{sec:hernilogika} poté vysvětluje vnitřní zapojení jednotlivých bloků zajišťujících logickou funkčnost hry. V sekci~\ref{sec:celek} je poté popsáno kompletní zapojení všech částí do funkčního celku.

\section{Modul pro čtení z PS/2 klávesnice}\label{sec:ctenizps2}
V této sekci vysvětluji číslicový obvod, který jako celek zpracovává vstupní signál z klávesnice a generuje daný výstup. Podsekce~\ref{sec:zpracovanivstupu} rozebírá číslicový obvod pro zpracování vstupního signálu a podsekce~\ref{sec:generovanivystupu} řeší číslicový obvod pro generování výstupního signálu.
\subsection{Zpracování vstupu}\label{sec:zpracovanivstupu}
\subsection{Generování výstupu}\label{sec:generovanivystupu}

\section{Modul pro výstup na VGA monitor}\label{sec:vystupnavga}
Tato sekce popisuje číslicový obvod, který zařizuje funkčnost monitoru a možnost propsání výstupu na něj. V podsekci~\ref{sec:casovanivga} popisuji obvod, který zajišťuje funkčnost a v podsekci~\ref{sec:propsanitextur} vysvětluji, jak propisuji textury na monitor.
\subsection{Časování VGA}\label{sec:casovanivga}
\subsection{Propsání textur na monitor}\label{sec:propsanitextur}

\section{Herní logika}\label{sec:hernilogika}
V této sekci popisuji číslicový obvod, který zajišťuje vnitřní funkcionalitu samotné hry. V podsekci~\ref{sec:hernimapa} je vysvětlený obvod, který řeší herní plochu, na které se postavy pohybují. Podsekce~\ref{sec:ovladanipostav} popisuje obvod ovládající postavy ve hře a v podsekci~\ref{sec:cilehry} je vysvětlen obvod, který řeší splnění cílů hry.
\subsection{Herní mapa}\label{sec:hernimapa}
\subsection{Ovládání postav}\label{sec:ovladanipostav}
\subsection{Cíle hry}\label{sec:cilehry}

\chapter{Testování}

 \chapter*{Závěr}

\begin{thebibliography}{9}
\bibitem{Kubatova} 
\end{thebibliography}

 \addcontentsline{toc}{section}{Závěr}

\end{document}